\section{\module{robotparser} --- 
         Parser for robots.txt}

\declaremodule{standard}{robotparser}
\modulesynopsis{Accepts as input a list of lines or URL that refers to a
                robots.txt file, parses the file, then builds a
                set of rules from that list and answers questions
                about fetchability of other URLs.}
\sectionauthor{Skip Montanaro}{skip@mojam.com}

\index{WWW}
\index{World-Wide Web}
\index{URL}
\index{robots.txt}

This module provides a single class, \class{RobotFileParser}, which answers
questions about whether or not a particular user agent can fetch a URL on
the web site that published the \file{robots.txt} file.  For more details on 
the structure of \file{robots.txt} files, see
\url{http://info.webcrawler.com/mak/projects/robots/norobots.html}. 

\begin{classdesc}{RobotFileParser}{}

This class provides a set of methods to read, parse and answer questions
about a single \file{robots.txt} file.

\begin{methoddesc}{set_url}{url}
Sets the URL referring to a \file{robots.txt} file.
\end{methoddesc}

\begin{methoddesc}{read}{}
Reads the \file{robots.txt} URL and feeds it to the parser.
\end{methoddesc}

\begin{methoddesc}{parse}{lines}
Parses the lines argument.
\end{methoddesc}

\begin{methoddesc}{can_fetch}{useragent, url}
Returns true if the \var{useragent} is allowed to fetch the \var{url}
according to the rules contained in the parsed \file{robots.txt} file.
\end{methoddesc}

\begin{methoddesc}{mtime}{}
Returns the time the \code{robots.txt} file was last fetched.  This is
useful for long-running web spiders that need to check for new
\code{robots.txt} files periodically.
\end{methoddesc}

\begin{methoddesc}{modified}{}
Sets the time the \code{robots.txt} file was last fetched to the current
time.
\end{methoddesc}

\end{classdesc}

The following example demonstrates basic use of the RobotFileParser class.

\begin{verbatim}
>>> import robotparser
>>> rp = robotparser.RobotFileParser()
>>> rp.set_url("http://www.musi-cal.com/robots.txt")
>>> rp.read()
>>> rp.can_fetch("*", "http://www.musi-cal.com/cgi-bin/search?city=San+Francisco")
0
>>> rp.can_fetch("*", "http://www.musi-cal.com/")
1
\end{verbatim}
