\section{\module{aetypes} ---
         AppleEvent objects}

\declaremodule{standard}{aetypes}
  \platform{Mac}
%\moduleauthor{Jack Jansen?}{email}
\modulesynopsis{Python representation of the Apple Event Object Model.}
\sectionauthor{Vincent Marchetti}{vincem@en.com}


The \module{aetypes} defines classes used to represent Apple Event
object specifiers. An object specifier is essentially an address of an
object implemented in a Apple Event server. An Apple Event specifier
is used as the direct object for an Apple Event or as the argument of
an optional parameter. In AppleScript an object specifier is
represented by a phrase such as:
\code{character 23 of document "Semprini"}. The classes defined in
this module allow this specifier to be represented by a Python object
which is initialized as follows:
\code{res = Document(1).Character(23)}


The \module{AEObjects} module defines the following class:

\begin{classdesc}{ObjectSpecifier}{want, form, seld, from}
  This is the base class for representing object specifiers and is
  generally not constructed directly by the user. Its important
  functionality is to define an \function{__aepack__()} function,
  which returns the Apple Event descriptor containing the object
  specifier.  Its data members, set directly from the constructor
  arguments, are:
\end{classdesc}

\begin{memberdesc}{want}
  A four character string representing the class code of the
  object. These class codes are specified in Apple Event Suites; for
  example the standard code for a character object is the 4 bytes
  \samp{char}.
\end{memberdesc}
